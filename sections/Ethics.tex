\section{Ethics}
	\subsection{Grundbegriffe}
	
		\begin{longtable}{|p{0.2\textwidth}||p{0.75\textwidth}|}
			\hline
			Begriff
				& Erklärung\\
			\hhline{|=#=|}
			Sozialtheorie
				& Stellt sich die Frage "Wie sollen wir leben?".\\
			\hline
			Glückstheorie
				& Stellt sich die Frage "Was ist ein gutes, gelingendes, glückliches Leben?".\\
			\hline
			Handlungstheorie
				& Stellt sich die Frage "Was heisst verantwortungsvolles Handeln?".\\
			\hline
			Gerechtigkeitstheorie
				& Stellt sich die Frage "Was heisst Gerechtigkeit?".\\
			\hline
			Moral
				& Beschreibt das Normen- und Wertesystem, welches sich aus einer kulturellen Tradition entwickelt hat.\\
			\hline
			Ethik
				& Theorie, die sich mit der Moral beschäftigt. Sie fragt grundlegend nach einem gelingenden, glücklichen Leben.\\
			\hline
			Ethos
				& Gewohnheit, Sitte, Brauch bzw. Charakter, Seelenzustand\\
			\hline
			Deskriptive Methode
				& Hält die faktische Handlungs- und Verhaltensweise, die Wertvorstellungen und Geltungsansprüche einer Gesellschaft fest, ohne sie zu werten.\\
			\hline
			Normative Methode
				& Vorschreibendes Verfahren, welches universell gültige Regeln der Ethik sucht.\\
			\hline
			Relativismus
				& Bezieht seine Aussagen auf soziale, kulturelle, historische oder persönliche Gegebenheiten zurück.\\
			\hline
			Universalismus
				& Will universell gültige Aussagen machen, ohne dabei auf soziale, kulturelle, historische oder persönliche Unterschiede zu achten.\\
			\hline
			Metaphysik
				& Eine Grunddisziplin der Philosophie, welche die Fundamente, Voraussetzungen, Ursachen, Gesetzlichkeiten und Prinzipien, sowie Sinn und Zweck der gesamten Realität behandelt.\\
			\hline
			Kategorischer Imperativ
				& ''Handle nur nach derjenigen Maxime, durch die du zugleich wollen kannst, dass sie ein allgemeines Gesetz werde.'' - Immanuel Kant (Kant's grundlegendes Prinzip der Ethik)\\
			\hline
		\end{longtable}
		
	\subsection{Vergleich der behandelten Philosophen}
	
		\begin{longtable}{|p{0.1\textwidth}||p{0.2\textwidth}|p{0.2\textwidth}|p{0.2\textwidth}|p{0.2\textwidth}|}
			\hline
				& Aristoteles
				& Sokrates
				& Platon
				& Kant \\
			\hhline{|=#=|=|=|=|}
			Grundidee
				&
				&
				&
				& \\
			\hline
			Stand zur Religion
				&
				&
				& ''Ich behaupte, die Moral kommt vor jeder Religion''
				& \\
			\hline
			Höchste\newline Tugend
				& Weisheit
				&
				& Gerechtigkeit
				& Die Pflicht, seine Fähigkeit zu vernunftbestimmtem Handeln zu gebrauchen\\
			\hline
			Weitere wichtige Tugenden
				&	
					\begin{itemize}
						\item Gerechtigkeit
						\item Tapferkeit
						\item Mässigung
						\item Freigebigkeit
						\item Hilfsbereitschaft
						\item Sanftmut
						\item Wahrhaftigkeit
						\item Höflichkeit
						\item Einfühlsamkeit
					\end{itemize}
				&
				&
				& Kant vertritt keine Tugendethik, er behauptet, dass Tugenden nur in Begleitung des sittlich Guten nützlich sind. Mut als Tugend kann sowohl das Handeln eines Verbrechers, als auch eines Polizisten bestimmen. Der Kategorische Imperativ ist der Massstab.\\
			\hline
			Höchstes\newline Gut/Ziel
				& Glücklichsein. Er behauptet, dass dies erarbeitet werden kann.
				&
				&
				& Glückseligkeit, aber nur wenn wir sie für andere anstreben. \\
			\hline
			Wieso ethisch handeln?
				&
				& Wer unrecht tut, verliert die Selbstachtung, auch wenn es niemand erfährt.
				& Die innere Person (Seele) hat sich nach dem Guten und Gerechten einmal geschaut, und man muss sich danach ''nur'' daran orientieren. \textrightarrow\ Verlust der Orientierung.
				& Wer unrecht tut, verliert die Selbstachtung, auch wenn es niemand erfährt. \\
		\end{longtable}
